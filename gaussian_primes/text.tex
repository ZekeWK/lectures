\section{Gaussian Integers}

\begin{definition}[Gaussian integer]
  A complex number \(z \in \mathbb{C} \) is a Gaussian integer if 
  \[
    \Re (z), \Im(z)  \in \mathbb{Z}
  \]
  where \(\Re\) and \(\Im\) represent the real and imaginary parts respectively. Let 
  \[
    \mathbb{Z} \left[i\right]
  \]
  represent the set of all Gaussian integers.
\end{definition}

\begin{definition}[Norm]
  The norm of a Gaussian integer, \(z = a +bi\) where \(a, b \in \mathbb{Z} \), has norm
  \[
    N(z) = a^2 + b^2.
  \]
\end{definition}

\begin{problem}
  Calculate \(N\left(3 + i\right)\).
\end{problem}

\begin{problem}
  Let \(z, w \in \mathbb{Z} \left[i\right]\). Show that 
  \[
    N(z \cdot w) = N(z) \cdot N(w).
  \]
  Note! This property will be of the utmost importance in the next section.
\end{problem}

\begin{problem} Calculate \(N\left((3+i)  \cdot (4-i)\right)\).
\end{problem}

\begin{definition}[Divisibillity]
  A Gaussian integer, \(z\), is divisible by another Gaussian integer, \(w\), if there exists a third Gaussian integer, \(d\), such that
  \[
    z = d  \cdot w.
  \]
\end{definition}

\begin{problem}
  Is \(2\) divisible by any Gaussian integer \(z\) with \(1 < N(z) < 4\)?
\end{problem}
% \begin{corollary}[Rules of operations] 
%   Let \(z = a +bi\) and \( w = c + di\) with \(a, b, c, d \in \mathbb{Z} \). The following rules apply:
%   \begin{itemize}
%       \item \(z + w= a +bi + c + di = (a+c) + (b+d)i\) 
%       \item \(z  \cdot w = (a + bi) \cdot (c + di) = (ac - bd)\) 
%   \end{itemize}
% \end{corollary}

\section{Gaussian Primes}

\begin{definition}[Units]
  A unit is an element with a multilicative inverse. In the Gaussian integers the units are
  \[
    1, i, -1, -i.
  \]
  
\end{definition}

\begin{definition}[Gaussian prime]
  A non-unit Gaussian integer is a prime if it is only divisible by units or unit multiples of itself.
\end{definition}

% Add some minor problems before

\begin{problem}
  Prove that not all primes in \(\mathbb{Z} \) are primes in \(\mathbb{Z} \left[i\right]\).
\end{problem}

\begin{problem}
  Prove that all Gaussian integers with prime norms are Gaussian primes.
\end{problem}

\begin{theorem}[Fermat's theorem on sums of squares]
  A prime \(p > 2\) can be written as the sum \(a^2 + b^2\) where \(a, b \in \mathbb{Z} \) iff \(p \equiv 1 \mod 4\).
\end{theorem}

\begin{problem}
  Show that for all primes, \(p\), in \(\mathbb{Z} \) with \(p \equiv 1 \mod 4\), there exists a Gaussian integer with norm \(p\).
\end{problem}


\begin{problem}
  Prove that if \(p\) is a prime in \(\mathbb{Z} \) and \(p \equiv 3 \mod 4\), it is a Gaussian prime.
\end{problem}

\begin{corollary}[Categories of Gaussian primes]
  For each prime, \(p \), in the integers, it falls under one of the following categories: 
  \begin{itemize}
    \item If \(p \equiv 1 \mod 4\), then there exists a Gaussian prime, \(z\), such that \(N(z) = p\).
    \item If \(p \equiv 3 \mod 4\), then \(p\) is itself a Gaussian prime.
    \item If \(p = 2\), then \(p = (1+i)(1-i) \);
  \end{itemize}
\end{corollary}

\begin{theorem}[Euclid's Lemma for Gaussian Primes]
  Let \(p\) be a Gaussian prime and \(a, b, \in \mathbb{Z}[i] \). If \(p\) divides \( a \cdot b\) then \(p\) divides one of \(a\) or \(b\).
\end{theorem}

\begin{problem}
  Show that all Gaussian primes are in fact unit multples of the previously mentioned categories.
\end{problem}

\begin{theorem}[Lagrange's Lemma]
  If a prime in the integers, p, is congruent to 1 mod 4, then there exists an \(n\) such that \(p | n^2 + 1\).
\end{theorem}


\begin{problem}
  Without using anything proven by Fermat's theorem on sums of squares, prove it.
\end{problem}

\begin{problem}
  Show that if an integer can be written as the sum of two squares in more than one way, then it is not prime.
\end{problem}

% TODO! Use \mathbb{P}[i]

% \begin{problem}
%   We previously relied on the sum of squares theorem by fermat,
%   \[
%     p \equiv 1 \mod 4 \Leftrightarrow (\exists ~ a,b) ~~ p = a^2 + b^2
%   \]
%   for any prime \(p\). Therefore we will prove it using 
%   
% \end{problem}
% Add in euclids algorithm as well...
% Prove fermats theorem on usms of two squares using gaussian primes!!!

% We also need a problem with euclides theorem I think to make everything is contained?

% Make sure all logic here is contained, or at least is contained in my head.























% \section*{What should be taught?}
% \begin{itemize}
%   \item Gaussian integers?
%   \item Definition - Using divisibility in complex
%   \item Similarities to real primes?
%   \item Proofs how they're one to one mapping to primes?
%   \item Generation?
%   \item Some problems coupled to when we where generating them?
%   \item The big goal is to find the three cases under which they can all fall.
%   \item Maybe this should be the whole point of it. A very very small introduction, and then tackling these cases one by one?
%   \item The problems would be
%   \begin{itemize}
%     \item Prove that those with norm a prime must be primes
%     \item Prove that those in reals prime with congruence are primes in the complex
%     \item Prove that these are the only ones? % NEED to learn proof
%   \end{itemize}
%   \item The ways factors of two squares interact!
%   % NEED to learn congruence prime theorem
%   
% \end{itemize}
%
%
% The lecture should have the following definitions:
% Definition and repetion of basic aritmetic
% Working lightly with some Gaussian primes
% Proofs of when they are Gaussian primes
% Working somewhat harder with the Gaussian primes
