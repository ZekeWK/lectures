\section{Funktioner}

\begin{definition}[Funktion]
	En funktion är en kartläggning som för varje element i definitionsmängden, ger ett unikt värde i målmängden.

	\center
	\begin{tikzpicture}
		\begin{scope}[local bounding box=A, x=3cm, y=1cm]
			\node[minimum width=2em] (n-1-A) at (0,-0) {3};
			\node[minimum width=2em] (n-2-A) at (0,-1) {8};
			\node[minimum width=2em] (n-3-A) at (0,-2) {-6};
		\end{scope}
		\node[ellipse, draw, fit=(A), label={above:Definitionsmängd}] {};

		\begin{scope}[local bounding box=B, x=3cm, y=1cm]
			\node[minimum width=2em] (n-1-B) at (1,-0) {0.3};
			\node[minimum width=2em] (n-2-B) at (1,-1) {1.2};
			\node[minimum width=2em] (n-3-B) at (1,-2) {\(\pi\)};
		\end{scope}
		\node[ellipse, draw, fit=(B), label={above:Målmängd}] {};

		\draw[->] (n-1-A) -- (n-2-B);
		\draw[->] (n-2-A) -- (n-2-B);
		\draw[->] (n-3-A) -- (n-1-B);
	\end{tikzpicture}
\end{definition}










\begin{definition}[Funktion]
	En funktion är en kartläggning som för varje element i definitionsmängden, ger ett unikt värde i målmängden.
\end{definition}

\begin{exempel} \label{exempel:funktioner}
	Det finns många sorters funktioner, både i och utanför matematiken:
	\begin{itemize}
		\item \(f(x) = x^2\)
		\item \(g(n) = \begin{cases} 0, \text{ n udda}  \\ 1, \text{ n jämn} \end{cases}\)
		\item \(u(x, t) = \text{``Värmen i staven \(x\) cm från kanten vid tiden \(t\)''} \) 
		\item \(P(t, p) = \text{``Sannolikheten att tävlande \(t\) i en mattetävling löser problem \(p\) .''} \) 
		\item Funktionen som tar in en funktion, och ger dess invers.
	\end{itemize}
	Inte alla funktioner behöver vara möjliga att skriva med matematisk notation, eller ens vara lätta att beräkna. Men, de måste ge ett och samma svar om de får samma invärden.
\end{exempel}



\subsection{Viktiga Mängder} % Tror mycket här kan tas bort!

\begin{definition}[Definitionsmängd]
	Definitionsmängden till en funktion är mängden (samlingen) av alla tillåtna invärden.
\end{definition}

\begin{obs}
	Två olika funktioner kan ha samma regel, men olika definitionsmängder.
\end{obs}

\begin{definition}[Målmängd]
	Målmängden till en funktion är mängden (samlingen) av alla möjliga utvärden.
\end{definition}

\begin{definition}[Värdemängd]
	Värdemängden till en funktion är mängden (samling) av alla värden funktionen faktiskt antar.
\end{definition}

\begin{obs}
	Målmängden kan väljas till det som är smidigast, medan värdemängden är unikt bestämd.
\end{obs}


\begin{exempel}
	Funktionen \(f(x) = \sqrt{x-2}\) har oftast:
	\begin{itemize}
		\item Definitionsmängden: \([2, \infty)\) - ``Reella tal större än eller lika med 2''
		\item Målmängden: \(\mathbb{R}\) - ``Reella talen''
		\item Värdemängden: \([0, \infty ]\) - ``Icke-negativa reella talen''
	\end{itemize}
	men \(g(x) = \sqrt{x-2}\) kan istället ha:
	\begin{itemize}
		\item Definitionsmängden: Heltalen större än eller lika med 3
		\item Målmängden: \(\mathbb{R}\) - ``Positiva Reella talen''
		\item Värdemängden: \([0, \infty ]\) - ``Heltalen större än 0''
	\end{itemize}
	
\end{exempel}

\subsection{Viktiga egenskaper}

\begin{definition}[Injektiv]
	En funktion är injektiv om inga två element i definitionsmängden har samma värde i målmängden.
\end{definition}

\begin{obs}
	Injektiv innebär att om \(f(x) = f(y)\) så måste \(x = y\)
\end{obs}


\begin{definition}[Surjektiv]
	En funktion är surjektiv, om målmängden är samma som värdemängden.	
\end{definition}

\begin{obs}
	Surjektiv innebär att för varje \(y\), finns ett (eller flera) \(x\) så att \(f(x) = y\).
\end{obs}

\begin{definition}[Inverterbar]
	En funktion är inverterbar om den är både injektiv och surjektiv.
\end{definition}

\begin{definition}[Kontinuerlig (enklare definition)]
	En funktion är kontinuerlig om du kan gå längs grafen över hela definitionsmängden.	
\end{definition}

\begin{obs}
	Exakta definitionen är mer komplicerad, men det här är intuitionen.
\end{obs}

\begin{exempel} ~
	\center
	\begin{tikzpicture}
		\foreach[count=\i] \lseti/\lsetmi in {{A}/{1,3,5,-3},{B}/{5,12,8}} {
			\begin{scope}[local bounding box=\lseti, x=3cm, y=0.7cm]
				\foreach[count=\j] \lj in \lsetmi {
					\node[minimum width=2em] (n-\j-\lseti) at (\i,-\j) {\lj};
				}
			\end{scope}
			\node[ellipse, draw, fit=(\lseti), label={above:$\lseti$}] {};
		}
		\draw[->] (n-1-A) -- (n-2-B);
		\draw[->] (n-2-A) -- (n-1-B);
		\draw[->] (n-3-A) -- (n-3-B);
		\draw[->] (n-4-A) -- (n-1-B);
	\node at (current bounding box.north) [] {Surjektiv};
	\end{tikzpicture}
	\hfill
	\begin{tikzpicture}
		\foreach[count=\i] \lseti/\lsetmi in {{A}/{1,3,5},{B}/{5,12,8, 31}} {
			\begin{scope}[local bounding box=\lseti, x=3cm, y=0.7cm]
				\foreach[count=\j] \lj in \lsetmi {
					\node[minimum width=2em] (n-\j-\lseti) at (\i,-\j) {\lj};
				}
			\end{scope}
			\node[ellipse, draw, fit=(\lseti), label={above:$\lseti$}] {};
		}
		\draw[->] (n-1-A) -- (n-2-B);
		\draw[->] (n-2-A) -- (n-4-B);
		\draw[->] (n-3-A) -- (n-3-B);
	\node at (current bounding box.north) [] {Injektiv};
	\end{tikzpicture}
	\hfill
	\begin{tikzpicture}
		\foreach[count=\i] \lseti/\lsetmi in {{A}/{1,3,5},{B}/{5,12,8}} {
			\begin{scope}[local bounding box=\lseti, x=3cm, y=1cm]
				\foreach[count=\j] \lj in \lsetmi {
					\node[minimum width=2em] (n-\j-\lseti) at (\i,-\j) {\lj};
				}
			\end{scope}
			\node[ellipse, draw, fit=(\lseti), label={above:$\lseti$}] {};
		}
		\draw[->] (n-1-A) -- (n-2-B);
		\draw[->] (n-2-A) -- (n-1-B);
		\draw[->] (n-3-A) -- (n-3-B);
	\node at (current bounding box.north) [] {Inverterbar};
	\end{tikzpicture}
\end{exempel}
\subsection{Problem}

\begin{problem}
	Vad är definitionsmängden, målmängden, och värdemängden till funktionerna i exempel \ref{exempel:funktioner}?
\end{problem}

\begin{problem}
	Varför är en funktion inverterbar om den är både surjektiv och injektiv? Varför räcker det inte med en?
\end{problem}


% \begin{problem}
% 	Vilka av följande definierar funktioner?
% 	\begin{enumerate}
% 		\item \(f(x) = 1\) 
% 		\item \(f(x)^2 = x^2\) 
% 		\item \(f\) har dubbla värdet till dess invärde.
% 		\item \(f\) tar inte in veckodagen, och ger dagens väder.
% 	\end{enumerate}
% \end{problem}

\begin{problem}
	Låt
	\[
		f(x) = \sqrt{x + 1}.
	\]
	Vilka av följande kan vara \(g\):s definitionsmängd? Målmängd? Värdemängd?
	\begin{enumerate}
		\item Reella talen större än \(-1\) 
		\item Icke-negativa reella talen
		\item Reella talen
	\end{enumerate}
\end{problem}

\begin{problem}
	Låt \(f(x) = x^2\) vara en injektiv funktion med värdemängd alla positiva reella tal. Beskriv alla möjliga definitionsmängder! (Definitionsmängden behöver inte vara sammanhängande)
\end{problem}


% \begin{problem}
% 	Är \(f(x) = x^2\) injektiv eller surjektiv? Vad händer om vi låter definitionsmängden eller målmängden vara de positiva talen?
% \end{problem}

\begin{problem}
	Varför är \(f(x)\) definierad men inte \(g(x)\) om TODO! HMMMMMMM
	\begin{itemize}
		\item \(h(x) = x^2\),
		\item \(k(x) = x^3\) 
		\item \(f(x) = \sqrt{h(x)}\)  
		\item \(g(x) = \sqrt{k(x)}\)  
	\end{itemize}
\end{problem}

\begin{problem}
	Är en injektiv funktion av en injektiv funktion injektiv? Är en surjektiv funktion av en surjektiv funktion surjektiv?
\end{problem}

\begin{problem}
	Låt \(f(x)\) vara injektiv men inte surjektiv, och \(g(x)\) vara surjektiv men inte injektiv. Kan \(h(x) = g(f(x))\) vara inverterbar? Ge ett exempel eller motbevisa.
\end{problem}

\begin{problem}
	Hitta inversen till:
	\begin{itemize}
		\item \(f(x) = x^3\) 
		\item \(f(x) = 3x - 1\) 
		\item \(f(x) =\)  % TODO Skapa en svår!
	\end{itemize}
\end{problem}

\begin{problem}
	Hur ser grafen till en kontinuerlig injektiv funktion ut? Vad utmärker dem? Prata med en kamrat! Vad händer om den dessutom är surjektiv?
\end{problem}


\begin{problem}
	Visa att \(f(x) = \sum_{k=0} ^{101} (-x)^k\) är surjektiv! (Tips! använd att den är kontinuerlig)
\end{problem}

\begin{problem}
	Visa att \(f(x) = \sum_{k=0} ^{100} (-x)^k\) inte är surjektiv! (Tips! använd att den är kontinuerlig)
\end{problem}

\begin{problem}
	Visa att funktionen
	\[
		f(x) = x^{101} + \frac{1}{x} + \ln(x+3) + e^{x^2 + 2}
	\]
	definierad på de positiva reella talen, är inverterbar. (Tips! Rita den!)
\end{problem}

\begin{problem}
	Om \(f(x)\) är surjektiv, visa att \(f(x)\) är injektiv om
	\[
		g(x) = f(f(x))
	\]
	är injektiv.
\end{problem}

\begin{problem}
	Om \(f(x)\) är injektiv, visa att \(f(x)\) är surjektiv om
	\[
		g(x) = f(f(x))
	\]
	är surjektiv.
\end{problem}






TODO!
        - Finns det ett värde som får denna? (För olika funktioner)
        - Surjektiv, injektiv, på diskret mapping?
        - Varför är första ok, men inte den andra?
            - Funktioner som går in i varandra så att värdemängd ej matchar 
            - Funktioner som går in i varandra så att värdemängd matchar 
        - Surjektiv på surjektiv, är den surjektiv? Injektiv?
        - Injektiv på injektiv, är den surjektiv? Injektiv?
        - Injektiv, surjektiv på begränsade intervall?
        - Bestäm värdemängd av klurigt tudelat problem
        - Bestäm invers
        - Visa att det finns en invers






\section{Funktionalekvationer}
Ibland får man inte givet vad en funktion är, utom måste lista ut det! Det sker ofta i formen av funktionalekvationer.

\begin{exempel}
	Ett exempel på en funktionalekvation är att
	\[
		(x,y): f(x) + f(y) = x + y
	\]
	\textbf{för alla} värden på \(x, y\). Vi försöker alltså inte hitta värden på \(x, y\) som i en vanlig ekvation, utom vet att det gäller för alla, och vill hitta funktionen! 
	I den här funktionalekvationen, kan vi sätta in \(x=y\) (eftersom vi får välja fritt bland \(x, y\)), och får
	\[
		(x, -x) : f(x) + f(x) = x + x \Rightarrow f(x) = x.
	\]
\end{exempel}

\subsection{Metoder}
\begin{metod}[Substitution]
	Sätt in värden på parametrarna som ger bättre samband.
\end{metod}

\begin{metod}[Symmetri]
	Sätt in värden på parametrarna som gör om ena delen av ekvationen till den andra.
\end{metod}

\begin{metod}[Bestäm \(f(0)\)]
	Bestäm (om möjligt) värdet på \(f(0)\), och använd det för att se hur funktionen ändras därifrån.
\end{metod}

\begin{metod}
	Utnyttja injektivitet! Injektivitet gör att du kan ``skala bort'' funktionen, alltså att \(f(x) = f(x)\) ger \(x = y\).
\end{metod}

\begin{metod}
	Uttnyttja surjektivitet! Surjektivitet gör att du kan substituera in funktionen, ty du vet att det finns ett invärde \(\alpha \), så att \(f(\alpha ) = x\).
\end{metod}

\subsection{Problem}
\begin{problem}
	Bestäm \(f(x) : \mathbb{R} \rightarrow \mathbb{R} \) om
	\[
		2f(x) - f(-x) = x
	\]
\end{problem}

\begin{problem}
	Låt 
	\[
		f(x^2+ f(y)) = y + f(x^2).
	\]
	Visa att \(f(x)\) är:
	\begin{itemize}
		\item Surjektiv
		\item Injektiv
	\end{itemize}
\end{problem}

\begin{problem}
	
\end{problem}





\if 0
	- Problem
			- Lös
					-    f(x) + f(y) = x + y
					-    2f(x) - f(-x) = x

			- Visa att f, f(x^2+ f(y)) = y + f(x^2):
					- Är surjektiv
					- Är injektiv

			- Funktionen f är injektiv, bestäm den om
					- f(f(f(x))) = f(f()) ...
\fi




